\chapter{总结}

\section{全文总结}
本文研究了使用卷积神经网络对中文影评数据进行评分预测的问题。使用收集自豆瓣的中文影评进行了多组实验,比较和分析了卷积神经网络的不同结构对结果的影响。

本文首先对研究问题的背景和意义进行了简述;然后,本文对于情感分析的常用方法进行了介绍和分析,并对本文用作baseline实验的朴素贝叶斯分类器算法、支撑向量机算法的原理进行了详细描述,并整理出了它们的算法;之后,本文对神经网络模型在情感分析问题上的应用进行了说明,主要说明了神经网络的一般结构,词嵌入的相关概念和算法,重点对卷积神经网络及其在文本方面的应用进行了说明;本文详细描述了进行实验所使用的数据,以及各组实验的设置;最后,本文对实验的结果进行了汇报,并对结果进行了比较和分析。

从实验结果可以看出,基于卷积神经网络的方法在情感分析问题上有着不错的表现,使用宽度为1,2的卷积核的基于词语断句单层卷积神经网络模型超过了全部的baseline算法;其他结构的卷积神经网络也取得了很好的实验结果。

\section{未来工作展望}
在以后的工作中。首先可以考虑进一步对卷积神经网络的结构进行优化,以达到更好的实验效果,本文的实验中只对少数几个模型进行了实验。还有很大的调节空间;从实验结果中可以看出,LSTM模型在中文影评情感分类的问题上也有着很好的效果(在baseline算法中是最高的,也超过很多组非最优的卷积神经网络模型),而在文献\cite{tang2015document}中,Tang等人提出了一种将卷积神经网络与LSTM相结合的方法。未来的工作可以考虑将这个方法应用于我们的中文电影评论的评分预测。

第五章中分析了数据中噪音较多是影响实验结果的一个重要因素,未来可能考虑人工对数据进行一次筛选,去掉一些噪音。这样可以提供实验的准确度。

此外,评分预测并不一定要当作一个分类问题来解决。实际上,考虑到影评评分的五个分段并不是完全独立的,这个问题使用回归(regression)的方式来解决也有一定的道理。Drake等人在2008年提出过一种“情感回归”的方法\cite{drake2008sentiment}。未来的工作可以考虑先从支撑向量机回归模型开始,尝试用回归的方式解决这个问题。

