\chapter{实验结果与分析}
\section{实验结果}
Baseline实验结果如表格\ref{tab:results base}所示;卷积神经网络评分预测(五分类实验)实验结果如表格\ref{tab:results5}所示、二分类预测实验结果如表格\ref{tab:results2} 所示。

\begin{table}
\centering
\caption{Baseline实验结果} \label{tab:results base}
\begin{tabular}{c|c|c|c}
    \hline
      & Naive Bayes & SVM & LSTM \\
    \hline
    评分预测(五分类)实验 & 41.54\% & 38.7\% & 43.62\% \\
    \hline
    二分类实验 & 79.64\% & 78.31\% & 81.77\% \\
    \hline
\end{tabular}
\end{table}

\begin{table}
\centering
\caption{评分预测实验准确率结果} \label{tab:results5}
\begin{tabular}{c|c|c|c|c}
    \hline
    Filters Sizes & CNN\_char & CNN & CNN\_2\_channel & CNN\_2\_layer\\
    \hline
    1 & 40.41\% & 43.04\% & - & - \\
    \hline
    1,2 & 42.11\% & 44.12\% & 43.29\% & 40.61\% \\
    \hline
    1,2,3 & 42.49\% & 43.28\% & - & - \\
    \hline
\end{tabular}
\note{实验设置见第四章第二、第三节相关内容;双通道和双层只在单层的最优设置上进行了实验。}
\end{table}

\begin{table}
\centering
\caption{二分类实验结果} \label{tab:results2}
\begin{tabular}{c|c|c|c|c}
    \hline
    Filters Sizes & CNN\_char & CNN & CNN\_2\_channel & CNN\_2\_layer\\
    \hline
    1 & 77.89\% & 81.42\% & - & - \\
    \hline
    1,2 & 79.74\% & 82.05\% & 81.33\% & 77.29\% \\
    \hline
    1,2,3 & 80.47\% & 81.50\% & - & - \\
    \hline
\end{tabular}
\note{实验设置见第四章第二、第三节相关内容;双通道和双层只在单层的最优设置上进行了实验。}
\end{table}

\section{结果分析}
可以看到使用中文分词器分词的效果要好于按字断句,说明对中文语言文本分词的确可以提取出文本中很多的特征。而基于字断句的话,在训练数据不非常大的情况下(40000句仍然只能算比较少的数据量),或是模型复杂程度不够高的情况下,可能会导致一些信息或特征无法被提取出来,因而导致结果不如分词。但同时也可以注意到,在卷积核的设置包含了3或4这样比较大的窗口时,基于字断句的输入也会有很好的效果。这可能说明了说明卷积神经网络的确有提取局部特征(在这里体现为提取出字的组合的特征)的能力。

单层双通道的模型也取得了很好的实验结果,但其最好的结果没有超过单层单通道的最好实验结果。这与我们开始时的期望略有出入,可能是由于双通道的模型更容易过拟合。双层卷积层模型的结果也超过了baseline算法,但落后于单层卷积层的两种模型。这与图像数据领域的结果有很大差别,可能是因为文本数据每个单词相较于图像数据的像素,包含的信息要多得多。因此,即使在单层卷积层的模型也可以提取出很多的信息,也会有很好的结果。而多层之后,反而由于抽象的层数太高、过拟合的机会增大,可能会导致效果反而不如单层的网络。

另外也可以注意到,五分类问题的结果准确率普遍比较低,不超过45\%。这是由于对影评进行五分类本身是一个比较困难的问题,与本课题数据集相近的SST-1数据集(同样是影评数据,分为五类,分别为非常负面、负面、中立、正面和非常正面),目前最前沿的成果也只能达到48.7\%的准确率。这个问题准确率难以踢得很高是因为每个影评者自己的指标不同,同样一句影评,对于一些人来说会选择给四星,另一些人则会给五星、三星。此外,中间三档影评的区分度很低。本课题使用的数据集未经过人工筛选,只根据作者发过的影评数设置了一个阈值。故数据集中可能存在评论与打分不符,或是一些没有意义的垃圾信息。这些评论也会对结果产生影响。

\begin{table}
\centering
\caption{Confusion Matrix} \label{tab:confusion}
\begin{tabular}{c|c|c|c|c|c}
    \hline
     & 1 & 2 & 3 & 4 & 5\\
    \hline
    1 & 0 & 306 & 166 & 86 & 61\\
    \hline
    2 & 229 & 0 & 303 & 79 & 22 \\
    \hline
    3 & 280 & 687 & 0 & 992 & 340 \\
    \hline
    1 & 59 & 127 & 647 & 0 & 587\\
    \hline
    1 & 43 & 53 & 153 & 368 & 0\\
    \hline
\end{tabular}
\note{横坐标为Ground Truth,纵坐标为预测值。}
\end{table}

表格\ref{tab:confusion}是五分类问题最后结果的confusion matrix,即预测错误的分布。可以看到,大多数的错误分布在主对角线上,即出现在相邻的两个分段。尤其是中间二、三、四三个阶段,出现的错误非常多。这印证了我们上一段的分析。

因为五分类的结果准确率较低,难以应用在实际问题中。我们又增加了二分类的试验,可以看到,二分类实验的准确率很高,达到了82\%以上。

\section{小结}
本章对实验结果进行了总结和分析。比较了卷积神经网络的不同卷积核选取带来的结果差异;比较和分析了单层单通道和更复杂网络结构的实验结果;具体分析了为什么评分预测的准确率较低。

