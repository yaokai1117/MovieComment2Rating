\chapter{绪论}

\section{课题背景与意义}
近年来,随着各种社交媒体和论坛的发展,互联网上的影评已经成为了观影者表达观点、进行交流的重要方式。互联网上出现了许多以影评为主要内容的网站、社区,如IMDB,Rotten Tomatoes,以及中文的豆瓣等。在这些网站上每天都产生出大量的来自用户的文本。这些文本对于了解用户对于电影的评价、获得用户反馈、进行市场研究和进行推荐都有着很大的意义。

由于网站很少会对用户文本加以限制(除去字数限制以及对一些词的屏蔽外),分析这些文本并从中获取有用的信息远不如分析电影评分(点赞数、打星数)那样容易。如果要人工对这些评论进行统计分析,工作量会非常大。且很多网站的内容每天以非常快的速度增长,导致人工分析变得完全不可行。因此,使用算法对影评进行情感分析,使用算法从用户文本中预测该用户对电影的评分可以大大帮助网站分析来自于用户的数据。尤其是对于一些用户不必给出评分,可以只留下文本评论的网站,对影评进行情感分析可以从评论中挖掘出量化的信息。

【这里插一个评论与分数例子的表格】

情感分析的结果(可以是积极、消极的二分类结果,也可以是预测的评分),可以帮助网站了解用户对于某部电影的评价;由于每条文字影评都对应着具体用户,情感分析的结果也可以用于对用户群体的把握和研究,如作为推荐系统的输入。因此,对影评进行情感分析有着实际的意义。



\section{相关研究现状}
文本情感分析是指通过计算机技术对文本的观点、主客观性、情绪进行挖掘和分析,对文本的情感倾向做出分类和判断。

基于词库的文本情感分析方法会采用已知的情感词典(一般由人工筛选出常见的带有情感色彩的单词并进行标注),对于每一个句子,根据其包含的情感单词以及句子的语法结构计算该句的情感倾向。

现有的机器学习的方法被广泛应用于文本情感分析的问题。监督学习的方法如朴素贝叶斯、最大熵和支撑向量机等,都可以用于情感分析。Pang 等人(EMNLP 2002)使用这些方法在来自IMDB的英文影评数据上进行了二分类的实验,其结果证明使用单个单词(unigram)作为特征进行分类就可以取得很好效果,支撑向量机在3折交叉验证中取得了82.9\%的准确率。

非监督学习的方法也被应用于情感分析,Peter D. Turney等人(ACL  2002)使用一种基于情感极性(Sentiment Orientation)的非监督的方法。一个句子的分类由它包含的所有短语的情感方向的平均值决定。每个短语的情感方向由它与单词excellent和poor的相互信息(mutual information)决定。该方法在来自Epinions的410条评论上进行了验证,取得了74\%的准确率。此外,之前所述的基于情感词典的方法也属于无监督学习。

近年来,越来越多的深度学习模型被应用于情感分析。卷积神经网络(Convolutional Neural Network, 简称CNN)是一种最早发源于计算机图形学的深度学习模型(Krizhevsky等人, NIPS 2012)。Yoon Kim等人(EMNLP 2015)使用将单词embedding为词向量,再用word vector组成的矩阵来模拟图形矩阵的方法,将CNN应用在了文本分类的问题上。Yoon等人使用一个简单的只有一层卷积层的CNN在6个数据集上进行了实验,均取得了很好的效果。


\section{本课题主要内容}
本课题使用来自豆瓣的中文短评(每条长度不超过140字,带有评分,可以看作有标注的数据)作为语料,使用卷积神经网络等方法进行文本情感分析,由影评预测对应的评分。本课题主要做了以下工作:

【这里插一个流程图】

1.	获得数据。使用微软必应搜索的Cosmos分布式处理平台获取了来自豆瓣的16万短评数据以及10万长评数据。

2.	数据预处理。进行中文分词,使用Word2vec算法训练出中文单词的word embedding

3.	建立深度学习模型,调整网络结构,找到适合中文影评数据的模型;实现朴素贝叶斯等传统算法,作为比较。

4.	比较和分析实验结果

本文的组织如下:第一章为绪论,介绍选题的背景和意义,简述当前相关领域研究状况和本课题主要工作;第二章介绍了文本情感分析的相关定义,以及基于词库的方法和基于机器学习的方法;第三章介绍了深度学习的方法在文本情感分析问题上的应用,详细介绍了用于文本分类的卷积神经网络,包括其模型细节和具体算法;第四章介绍了使用卷积神经网络进行影评分类实验的数据集和实验设置;第五章列出了实验结果,并对baseline实验以及多组卷积神经网络的实验结果进行了对比和分析。
